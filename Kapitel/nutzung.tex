%!TEX root = ../hausarbeit.tex
\section{Welche Medien nutzen Jugendliche}

Die Ausstattung mit den unterschiedlichsten Medien hat in den letzten Jahren sowohl in privaten Haushalten als auch in der Schule deutlich zugenommen. Meilensteine waren dabei der Computer/Laptop, das Smartphone, die Spielekonsole und natürlich das Internet. Die Nutzung der Medien ist für die meisten Jugendlichen heutzutage vollständig in den Alltag integriert. \todo{schreiben, dass es mehrere Studien gibt, ein paar aufzählen und dann sagen, dass man sich, um den Rahmen nicht zu sprengen, auf die Jim Studie begrenzt...}

Die JIM Studie\footnote{JIM (Jugend, Information, (Multi-)Media) ist eine Basisstudie zum Medienumgang 12- bis 19-Jähriger in Deutschland. Herausgeber ist der Medienpädagogische Forschungsverbund Südwest unter der Leitung von Peter Behrens und Thomas Rathgeb} bietet repräsentative und objektive Daten und Fakten rund um den aktuellen Medienumgang von 12- bis 19-Jährigen. Dafür wurde in der Zeit vom 7. Mai bis 17. Juni 2012 eine repräsentative Stichprobe von 1.201 Jugendliche im Alter von 12 bis 19 Jahren telefonisch befragt. Es nahmen 49\% Mächen und 51\% Jungen teil die zu 86\% Schüler waren und zu 54\% das Gymnasium besuchten. 

In \citet{jim12} werden folgende Daten zur akteullen Mediennutzung angegeben: 100\% der Jugendlichen gaben an, dass in ihrem Haushalt ein Laptop/Computer vorhanden ist. 79\% der Mädchen und 85\% der Jungen gaben an, einen eigenen Laptop/Computer zu besitzen. In 98\% der Fälle ist ein Handy, ein Fernseher und Internetzugang im Haushalt vorhanden, 98\% der Mädchen und 95\% der Jungen besitzen ein eigenes Handy (davon entfallen 47\% auf ein Smartphone), 55\% der Mädchen und 64\% der Jungen verfügen über einen eigenen Fernseher und 85\% der Mädchen und 88\% der Jungen können von ihrem Zimmer aus auf das Internet zugreifen. 69\% gaben an, dass sie feste und tragbare Spielekonsolen im eigenen Haushalt haben, davon besitzen 45\% der Mädchen und 57\% der Jungen eigene Konsolen, die sie nicht mit der Familie teilen müssen. 